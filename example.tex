\documentclass[a4paper,10pt]{article}
\usepackage[utf8x]{inputenc}
\usepackage[pdftex]{graphicx}
\usepackage{graphviz}



% using dot2tex converted output
\usepackage{tikz}
\usetikzlibrary{arrows,shapes}
\usepackage{amsmath}

% this will place all graphviz files into subdirectory "graphviz" which
% must already exist!
% \def\graphvizDir{graphviz}

\begin{document}

\begin{figure}
  \begin{digraph}{graph1}
               a -> b -> c;
               a -> {x y};
               b [shape=box];
               c [label="hello\nworld",color=blue,fontsize=24,
                    fontname="Palatino-Italic",fontcolor=red,style=filled];
               a -> z [label="hi", weight=100];
               x -> z [label="multi-line\nlabel"];
               edge [style=dashed,color=red];
               b -> x;
               {rank=same; b x}
  \end{digraph}
 \caption{Using default \texttt{dot} filter to place nodes}
\end{figure}

\begin{figure}
  \begin{digraph}[scale=0.5,layout=neato,dir=graphviz]{graph2}
      a -> b;
      c -> b;
      d -> b;
      e -> b;
  \end{digraph}
 \caption{Using \texttt{neato} filter to place nodes, output will be stored in \texttt{graphviz} directory}
\end{figure}

\begin{figure}
  \begin{digraph}[scale=0.6]{graph3}
  root [ shape=Mrecord label="<f1> A|<f2> B|{<f3> C|<f4> D}|<f5> E" ];
  \end{digraph}
  \caption{testing shapes, \texttt{scale=0.5}}
\end{figure}

\begin{figure}
  \begin{digraph}[layout=twopi,scale=0.6]{un-safe-name}
      a -> b;
      c -> b;
      d -> b;
      e -> b;
  \end{digraph}
 \caption{Using \texttt{twopi} filter to place nodes}
\end{figure}

\begin{figure}
  \includedot{test}
 \caption{\texttt{includedot} command test}
\end{figure}

\begin{figure}
  \begin{digraph}[scale=0.5,output=png]{graph5}
      node [shape=record, height=.1];
      "*"[label="<l> |<h> *|<p> "];
      "3" [label="<l> |<h> 3|<p> "];
      "+" [label="<l> |<h> +|<p> "];
      "4" [label="<l> |<h> 4|<p> "];
      "5" [label="<l> |<h> 5|<p> "];
      "*":l -> "3":h;
      "*":p -> "+":h;
      "+":l -> "4":h;
      "+":p -> "5":h;
  \end{digraph}
 \caption{Output to PNG format}
\end{figure}

\begin{figure}
  \begin{digraph}[output=tex]{graph6}
      c -> b;
      d -> b;
      e -> b;
  \end{digraph}
  \caption{digraph with option \texttt{output=tex}, \texttt{dot2tex} is used for conversion}
\end{figure}

\begin{figure}
  \includedot[output=tex]{test2}
 \caption{\texttt{includedot} command test}
\end{figure}

\begin{figure}
  \begin{graphviz}[scale=0.5]{graph7}
      a -- b;
      a -- c;
      b -- c;
  \end{graphviz}
  \caption{non-oriented diagram produced in \texttt{graph} environment}
\end{figure}

\begin{figure}
  \begin{digraph}[scale=0.8]{graph8}
    /* declare nodes & style them */
    a [label="Node 1"];
    b [label="Node 2",style=filled,fillcolor=coral ];
    c [label="Node 3", shape=square, style=filled, fillcolor=green ];
    d [label="Node 4", style=filled, fillcolor=cadetblue1,peripheries = 2  ]
    /* declare edges & style them */
    a -> b [dir=none, weight=1, penwidth=3] ;
    b -> c [dir=both, color=maroon1] ;
    a -> d [arrowsize=1.5, weight=2.]
  \end{digraph}
  \caption{some colors and shapes}
\end{figure}

\end{document}
