\documentclass[a4paper,10pt]{article}
\usepackage[utf8x]{inputenc}
\usepackage[pdftex]{graphicx}
\usepackage{graphviz}

\begin{document}

\begin{figure}
  \digraph{graph1}{%
               a -> b -> c;
               a -> {x y};
               b [shape=box];
               c [label="hello\nworld",color=blue,fontsize=24,
                    fontname="Palatino-Italic",fontcolor=red,style=filled];
               a -> z [label="hi", weight=100];
               x -> z [label="multi-line\nlabel"];
               edge [style=dashed,color=red];
               b -> x;
               {rank=same; b x}
  }
 \caption{Using default \texttt{dot} filter to place nodes}
\end{figure}


\begin{figure}
  \digraph[scale=0.5,command=neato]{graph2}{%
     a -> b;
     c -> b;
     d -> b;
     e -> b;
  }
 \caption{Using \texttt{neato} filter to place nodes}
\end{figure}

  \digraph[scale=0.5]{graph3}{%
      root [ shape=Mrecord label="<f1> A|<f2> B|{<f3> C|<f4> D}|<f5> E" ];
  }


\begin{figure}
  \digraph[scale=0.5,command=twopi]{graph4}{%
     a -> b;
     c -> b;
     d -> b;
     e -> b;
  }
 \caption{Using \texttt{twopi} filter to place nodes}
\end{figure}

\begin{figure}
  \includedot{test}
 \caption{\texttt{includedot} command test}
\end{figure}

\begin{figure}
  \digraph[scale=0.5,output=png]{graph5}{%
	node [shape=record, height=.1];
	"*"[label="<l> |<h> *|<p> "];
	"3" [label="<l> |<h> 3|<p> "];
	"+" [label="<l> |<h> +|<p> "];
	"4" [label="<l> |<h> 4|<p> "];
	"5" [label="<l> |<h> 5|<p> "];
	"*":l -> "3":h;
	"*":p -> "+":h;
	"+":l -> "4":h;
	"+":p -> "5":h;
  }
 \caption{Output to PNG format}
\end{figure}




\end{document}
